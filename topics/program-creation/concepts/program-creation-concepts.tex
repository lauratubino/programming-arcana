% ============
% = Concepts =
% ============
\clearpage
\section{Program Creation Concepts} % (fold)
\label{sec:program_creation_concepts}

Our first program is going to display some text to the Terminal. In this section you will be introduced to the programming artefacts and terminology you will need to use to create this program. This first step is important and will require you to have installed a C++ or Pascal compiler, see \cref{cha:building programs} \nameref{cha:building programs} for instructions.

A programming \textbf{artefact} is something that can be created and used within your code. In this chapter we will look at creating programs, and using a number of other artefacts. The following artefacts will be covered in this chapter:
\begin{itemize}
  \item \nameref{sub:program}: A program is a sequence of instructions that when compiled creates an executable file that a user can run.
  \item \nameref{sub:procedure}: A procedure is a named sequence of instructions that will get the computer to perform a task. When you want the task performed you can call the procedure.
  \item \nameref{sub:library}: The program can use code from other Libraries. These libraries contain reusable Procedures and Types. 
  \item \nameref{sub:type}: A type defines how data is interpreted by the program. The programming language will support a number of basic types by default, and libraries can add other types. 
\end{itemize}

In addition to these artefacts, you will need to understand some programming \textbf{terminology}. The following terms are discussed in this section:
\begin{itemize}
  \item \nameref{sub:statement}: An \textbf{instruction} within the program.
  \item \nameref{sub:expression}: A \textbf{value} used in a statement.
  \item \nameref{sub:identifier}: The \textbf{name} of an artefact.
  % \item Literal: A part of an \textbf{expression} where the value is entered directly into the code.
\end{itemize}

This section also introduces the following kinds of instructions. You can use these to get the computer to perform certain \textbf{actions} within your program.
\begin{itemize}
  \item \nameref{sub:procedure call}: The instruction to run a procedure.
\end{itemize}

We can then use these concepts, artefacts, and instructions to create a program that will write some text to the Terminal as shown in Figure \ref{fig:program-creation-helloworld}.

\begin{figure}[h]
   \centering
   \includegraphics[width=0.8\textwidth]{./topics/program-creation/images/HelloWorld} 
   \caption[Hello World Terminal]{Hello World run from the Terminal}
   \label{fig:program-creation-helloworld}
\end{figure}

% \clearpage
% \subsection{C++ Program} % (fold)
% \label{sub:program_in_c}

% The C++ programming language does not have an explicit Program artefact for you to create. Rather, in C++ a program is implied by the existence of a special function called `\texttt{main}' somewhere in your source code. Figure \ref{csynt:program-creation-program} shows the structure of the syntax you can use to create a program using the C++ language.

% \csyntax{csynt:program-creation-program}{a Program}{program-creation/program}

% The code in Listing \ref{lst:program-creation-c-hello-world} shows an example C++ Program. You should be able to match this up with the syntax defined in Figure \ref{csynt:program-creation-program}. Notice at the start of the code the syntax indicates we can have an optional \emph{header include}, this matches up with the first line in the code where it \emph{includes} the `splashkit.h' header file. Declaration of the \texttt{main} function follows the inclusion of the header file, and it contains the instructions that are executed when the program runs.

% \csection{\ccode{lst:program-creation-c-hello-world}{C++ Hello World}{code/c/program-creation/hello-world.c}}

% \mynote{
%   \begin{itemize}
%     \item When a C++ \nameref{sub:program} runs, it start running the instructions from the first \nameref{sub:statement} within the \texttt{main} function (line 5).
%     \item A \nameref{sub:function} is a kind of \nameref{sub:procedure}, and their details will be covered later (see \sref{sub:function}).
%     \item The `\texttt{return 0}' code is a \nameref{sub:statement} that ends the \texttt{main} function (and the program). The \nameref{sub:return_statement} is covered later in \sref{sub:return_statement}.
%     \item With the \emph{header include} syntax you use \csnipet{#include <...>} to include standard libraries, and \csnipet{#include "..."} to include other external libraries.
%     \item Header files contain a summary of the features available within a library. By including the header file you gain access to these features.
%   \end{itemize}
% }

% % subsection program_in_c (end)
\clearpage
\subsection{C++ Statement} % (fold)
\label{sub:program-creation-c_statement}

In a \nameref{sub:statement} you are commanding the computer to perform an action. There are only a small number of statements you can choose from. At this stage the only statement is the \nameref{sub:procedure call}, technically the \emph{procedure statement} in C++. This is shown in Figure \ref{csynt:program-creation-statement}, where we can see that at this stage all Statements are calls to \nameref{sub:procedure}s.

\csyntax{csynt:program-creation-statement}{Statement Syntax}{program-creation/statement}

\csection{\ccode{lst:program-creation-c-knights}{C++ Knights}{code/c/program-creation/knights.c}}

\mynote{
\begin{itemize}
  \item The code in Listing \ref{lst:program-creation-c-knights} contains a \nameref{sub:program_in_c}.
  \item This Program contains five procedure calls, see \nameref{sub:program-creation-c_procedure_call}.
  \item Each procedure call runs the \csnipet{write_line} or \csnipet{write} procedure to output text to the Terminal. See the section on \nameref{sub:c_console_output}.
\end{itemize}
}

% subsection c_statement (end)
\clearpage
\subsection{C++ Procedure Call} % (fold)
\label{sub:program-creation-c_procedure_call}

A procedure call allows you to run the code in a Procedure, getting its instructions to run before control returns back to the point where the procedure was called.

\csyntax{csynt:program-creation-procedure-call}{Procedure Call Syntax}{program-creation/procedure-call}

\csection{\ccode{lst:program-creation-c-count-back}{C++ Count Back}{code/c/program-creation/count-back.c}}

\mynote{
\begin{itemize}
  \item A procedure call is an \textbf{action} which commands the computer to run the code in a procedure.
  \item The procedure call starts with the procedure's name (its \nameref{sub:identifier}) that indicates the procedure to be called.
  \item Following the identifier is a list of values within parenthesis, these are the values (coded as \nameref{sub:expression}s) that are passed to the procedure for it to use.
  \item Remember that C++ is case sensitive so using \csnipet{Write_Line} instead of \csnipet{write_line} will not work.
  \item The code in \lref{lst:program-creation-c-count-back} contains a \nameref{sub:program_in_c}.
  \item This Program contains four procedure calls.
  \item Each procedure call runs the \csnipet{write_line} procedure to output text to the Terminal. See the section on \nameref{sub:c_console_output}.
  
\end{itemize}
}


% subsection c_procedure_call (end)
\input{topics/program-creation/concepts/procedure}
\clearpage
\subsection{C++ Expression} % (fold)
\label{sub:program-creation-c_expression}

An \nameref{sub:expression} in C++ is a mathematical calculation or a \nameref{sub:program-creation-c_literal} value. Each expression will have a \nameref{sub:type}, and can contain a number of mathematic operators. Table \ref{tbl:program-creation-c operators and expresions} lists the operators that you can include in your expressions, listed in order of precedence.\footnote{Expressions follow the standard mathematic order of precedence (BODMAS).} The operators you can use depend on the kind of data that you are using within the expression.

\begin{table}[h]
  \begin{minipage}{\textwidth}
  \centering
  \begin{tabular}{|c|l|l|}
    \hline
    \textbf{Operator} & \textbf{Description} & \textbf{Example} \\
    \hline
    \texttt{ ( ) }     &   Parenthesis                 & \texttt{(1 + 1) * 2}  \\
    \texttt{\% * /}      &   Modulo\footnote{The remainder after division. For example 9 modulo 3 is 0, 10 modulo 3 is 1, 11 modulo 3 is 2 etc.}, Multiplication and Division & \texttt{1 / 2 * 5 \% 3}    \\
    \texttt{+ -}      &   Addition and subtraction    & \texttt{10 + 3 - 4}   \\
    \hline
  \end{tabular}
  \end{minipage}
  \caption{C++ Operators and Example Expressions}
  \label{tbl:program-creation-c operators and expresions}
\end{table}

\begin{table}[h]
  \begin{minipage}{\textwidth}
  \centering
  \begin{tabular}{|c|c|l|}
    \hline
    \textbf{Example Expression} & \textbf{Value} & \textbf{Type} \\
    \hline
    \texttt{ 73 }     &   73                 & \texttt{int}  \\
    \texttt{ 2.1 }      & 2.1   & \texttt{float}    \\
    \texttt{ "Hello World" }      &   "Hello World"    & \texttt{string}\footnote{This is technically a \texttt{char*} which denotes a reference to a \textit{string} of characters.}   \\
    \texttt{ "Fred" }      &   "Fred"    & \texttt{string}   \\
    \texttt{ 3 * 2 } & 6 & \texttt{int} \\
    \texttt{ 1 + 3 * 2 }  & 7 & \texttt{int} \\
    \texttt{ (1 + 3) * 2} & 8 & \texttt{int} \\
    \texttt{ 7 - 3 + 1 }  & 5 & \texttt{int} \\
    \texttt{ 3 / 2 } & 1\footnote{C does integer division for int values, rounding the value down.} & \texttt{int} \\
    \texttt{ 3.0 / 2.0} & 1.5 & \texttt{float} \\
    \texttt{ 3 \% 2} & 1 & \texttt{int} \\
    \texttt{ 11 \% 3} & 2 & \texttt{int} \\
    \texttt{ 3 / 2.0 } & 1.5\footnote{If either, or both, values are real (floating point) numbers the result is also a real number.} & \texttt{float} \\
    \texttt{ 1 + (3 / 2.0) + 6 * 2 - 8} & 6.5 & \texttt{float} \\
    \hline
  \end{tabular}
\end{minipage}
  \caption{Example C++ Expressions and their values}
  \label{tbl:program-creation-c example expresions}
\end{table}

\mynote{
\begin{itemize}
  \item Table \ref{tbl:program-creation-c example expresions} shows some example expressions, their values, and types.
  \item Expressions can be literal values, entered in the code.
  \item Expression can contain mathematical calculations using standard addition, subtraction, multiplication, division, and groupings.
\end{itemize}
}


% subsection c_expression (end)
\clearpage
\subsection{C++ Literal} % (fold)
\label{sub:program-creation-c_literal}

A literal is either a number or text value written directly in the code. In other words, it is not \emph{calculated} when the program runs - the value entered is \textbf{literally} the value to be used. Figure \ref{csynt:program-creation-literal} shows the syntax for the different literal values you can enter into your C++ code.

\csyntax{csynt:program-creation-literal}{Literals}{program-creation/literal}

\mynote{
\begin{itemize}
  \item Most of these are self evident. For example, a whole number can be \csnipet{127}, \csnipet{-8711}, \csnipet{+10} for example. Real numbers are written as \csnipet{3.1415} for example.
  \item Within a string the {\textbackslash} character is used to indicate that the next character has a special meaning. The following list includes the most useful special characters:
  \begin{itemize}
    \item \texttt{{\textbackslash}n} creates a new line
    \item \texttt{{\textbackslash}"} creates a double quote
    \item \texttt{{\textbackslash}\%} creates a \% character
    \item \texttt{{\textbackslash}{\textbackslash}} creates a {\textbackslash}
  \end{itemize}
  % \item `0..9' means the digits 0, 1, 2, etc. up to 9.
  % \item `\emph{any character except ", {\textbackslash}, \%, or new line}' allows you to include any character, with those that can not be includes directly being able to be included using the escape sequence (e.g. {\textbackslash}n for new line).
\end{itemize}
}

% subsection c_literal (end)
\clearpage
\subsection{C++ Types} % (fold)
\label{sub:program-creation-c_types}

\nameref{sub:type}s are used to define how data is interpreted and the operations that can be performed on the data. Table \ref{tbl:program-creation-c-types} shows the three basic types of data, the associated C++ type, size in memory, and other related information. Table \ref{tbl:program-creation-c operators by type} shows the operators that are permitted for each Type.

\begin{table}[h] 
\begin{minipage}{\textwidth}
\centering
\begin{tabular}{|l|c|c|c|}
\hline
\multicolumn{4}{|c|}{\textbf{Whole Number Types}} \\
\hline
\emph{Name} & \emph{Size} & \multicolumn{2}{|c|}{\emph{Range (lowest .. highest)}} \\
\hline
\csnipet{short} & 2 bytes/16 bits & \multicolumn{2}{|c|}{-32,767 .. 32,767} \\
\csnipet{int} & 4 bytes/32 bits & \multicolumn{2}{|c|}{-2147483648 .. 2147483647} \\
\csnipet{int64_t}    & 8 bytes/64 bits & \multicolumn{2}{|c|}{-9,223,372,036,854,775,807 ..} \\
  & & \multicolumn{2}{|c|}{9,223,372,036,854,775,807} \\
\hline
\multicolumn{4}{c}{} \\
\hline
\multicolumn{4}{|c|}{\textbf{Real Number Types}} \\
\hline
\emph{Name} & \emph{Size} & \emph{Range (lowest .. highest)} & \emph{Significant Digits} \\
\hline
\csnipet{float} & 4 bytes/32 bits & 1.0e-38 .. 1.0e38 & 6 \\
\csnipet{double} & 8 bytes/64 bits & 2.0e-308 .. 2.0e308 & 10 \\
\hline
\multicolumn{4}{c}{} \\
\hline
\multicolumn{4}{|c|}{\textbf{Text Types}} \\
\hline
\emph{Name} & \emph{Size} & \multicolumn{2}{|c|}{\emph{Known As}} \\
\hline
\csnipet{char}  & 1 byte/8 bits & \multicolumn{2}{|c|}{} \\
\hline
\csnipet{string} & various\footnote{The size in memory is determined by the number of characters within the string, and some overhead} &  \multicolumn{2}{|c|}{c-string} \\
\hline
\end{tabular}
\caption{C++ Data Types}\label{tbl:program-creation-c-types}
\end{minipage}
\end{table}

\begin{table}[h]
  \begin{minipage}{\textwidth}
  \centering
  \begin{tabular}{|c|c|l|}
    \hline
    \textbf{Type} & \textbf{Operations Permitted} & \textbf{Notes}\\
    \hline
    Whole Numbers     &   \texttt{( ) + - / * \%} & Division rounds down if all\\
                                    &                        & values are whole numbers.\\
    Real Numbers   &   \texttt{( ) + - / *} &    \\
       & & \\
    Text           &   \texttt{( ) +}          & You can use \texttt{+} for concatenation.\footnote{To concatenate literals you \textbf{must} tell the compiler to make them strings. This can be done using \csnipet{string("...")}. See \fref{csynt:program-creation/typed-literal}.} \\
    \hline
  \end{tabular}
  \caption{C++ Permitted Operators by Type}
  \label{tbl:program-creation-c operators by type}
\end{minipage}
\end{table}

\csyntax{csynt:program-creation-typed-literal}{Typed Literals}{program-creation/typed-literal}

\mynote{
\begin{itemize}
  \item The \texttt{int} type is the typical whole number type.
  \item The \texttt{double} type is the typical real number type.
  \item C++ builds on the C programming language, and inherits a the `c-string' standard literal. This can be converted to a C++ string, but needs you to indicate this as shown. You cannot perform concatenation on the c-string type, but you can on C++ strings.
  \item For example values see Table \vref{tbl:program-creation-c example expresions}.
\end{itemize}
}
% subsection c_types (end)
\clearpage
\subsection{C++ Identifier} % (fold)
\label{sub:c_identifier}

The C++ \nameref{sub:identifier} syntax is shown in Figure \ref{csynt:program-creation-identifier}. In C++, as in most programming languages, the identifier must start with an underscore (\_) or a letter; in other words your identifiers cannot start with a number or contain other symbols. This is because the compiler needs a way of distinguishing identifiers from numbers entered directly into the code.

\csyntax{csynt:program-creation-identifier}{an Identifier}{program-creation/identifier}

\begin{table}[h]
  \centering
  \begin{tabular}{|ccccc|c|cc|}
    \hline
    \multicolumn{5}{|c|}{\textbf{Reserved Identifiers (Keywords)}} & & \multicolumn{2}{c|}{\textbf{Example Identifiers}} \\
    \hline
    \texttt{auto}     &   \texttt{break}    & \texttt{case}     &   \texttt{char}     &   \texttt{const}  & & printf & scanf  \\         
    \texttt{continue} &   \texttt{default}  &  \texttt{do}      &   \texttt{double}   &   \texttt{else}   & & bitmap & sound\_effect  \\
    \texttt{enum}     &   \texttt{extern}   & \texttt{float}    &   \texttt{for}      &   \texttt{goto}   & & name & draw\_bitmap  \\
    \texttt{if}       &   \texttt{int}      &   \texttt{long}   &   \texttt{register} &   \texttt{return} & & age & my\_alien \\         
    \texttt{short}    &   \texttt{signed}   & \texttt{sizeof}   &   \texttt{static}   &   \texttt{struct} & & height & test  \\          
    \texttt{switch}   & \texttt{typedef}  &   \texttt{union}    &   \texttt{unsigned} &   \texttt{void}   & & alien & name3 \\
    \texttt{volatile} &   \texttt{while}    &  & & &                                                        & \_23  & i \\
    \hline
  \end{tabular}
  \caption{C++ Keywords and other example identifiers}
  \label{tbl:program-creation-c identifiers and keywords}
\end{table}

\mynote{
\begin{itemize}
  \item In the syntax definition an identifier cannot contain spaces, or special characters other than underscores (\_).
  \item A letter is any alphabetic character (\emph{a} to \emph{z} and \emph{A} to \emph{Z}).
  \item A digit is a single number (\emph{0} to \emph{9}).
  \item Each item in Table \ref{tbl:program-creation-c identifiers and keywords} is a valid identifier.
  \item The \textbf{keywords} are identifier that has special meaning to the language.
  \item The \textbf{example identifiers} give you examples of the kinds of names that could be given to artefacts we create.
\end{itemize}
}

% subsection c_identifier (end)
\input{topics/program-creation/concepts/library}
\input{topics/program-creation/concepts/comments}
\clearpage
\subsection{Procedure Declarations} % (fold)
\label{sub:proc_decl-procedure_declarations}

Procedures contain code that define the steps the computer performs when the procedure is called. In your Program you can define your own Procedures, allowing you to divide a program's tasks into separate Procedures.

\begin{figure}[h]
   \centering
   \includegraphics[width=\textwidth]{./topics/program-creation/diagrams/ProcedureDeclaration} 
   \caption{Procedure Declaration}
   \label{fig:procedure-decl-procedure-decl}
\end{figure}

\mynote{
\begin{itemize}
  \item A Procedure is an \textbf{artefact} that you can \emph{create} and \emph{use} in your code.
  \item Each Procedure contains code to perform a certain task. When you want the task performed you call the Procedure.
  \item Procedures should have a \textbf{side effect}\footnote{Output to the Terminal is an example of a Side Effect. After calling these procedures the text you wanted to appear was written to the Terminal. These Procedures changed the Terminal.}, meaning that it changes something when it is executed.
  \item The Procedure's declaration defines its \textbf{name}, and the \textbf{steps} it performs.
  \item Each instructions in the Procedure is a \nameref{sub:statement}.
  \item The Procedure's \nameref{sub:identifier}:
  \begin{itemize}
    \item Is the name used to call the Procedure.
    \item Should be a \textbf{verb} that \textbf{reflects the task} the Procedure performs.
  \end{itemize} 
  \item When the Procedure is called its instructions are executed.
  \item Each Procedure's instructions are isolated from the other code in your Program. When you are working on a Procedure you do not need to know about the internal workings of the other procedures.
\end{itemize}
}

% subsection procedure_declarations (end)

% section program_creation_concepts (end)

\clearpage
\subsection{Summary} % (fold)
\label{sub:program_creation_concepts_summary}

This section has introduced a number of programming artefacts, some programming terminology, and one kind of instruction. An overview of these concepts is shown in Figure \ref{fig:program-creation-summary}. The next section will look at how you can use these concepts to design some small programs.

\begin{figure}[h]
   \centering
   \includegraphics[width=\textwidth]{./topics/program-creation/diagrams/Summary} 
   \caption[Chapter Concepts]{Key Concepts introduced in this Chapter}
   \label{fig:program-creation-summary}
\end{figure}

\mynote{
\begin{itemize}
  \item \textbf{Artefacts} are things you can \emph{create} and \emph{use}.
  \item \textbf{Terms} are things you need to \emph{understand}.
  \item \textbf{Actions} are things you can \emph{command} the computer to perform.
\end{itemize}
}

% subsection summary (end)
